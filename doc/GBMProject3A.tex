\documentclass{article}

\usepackage{graphicx}
\usepackage{amsmath}
\usepackage{amssymb}
\usepackage{gensymb}

\begin{document}

\title{Documentation de la suite logicielle GBMProject3A}
\author{Guillaume Gibert}
\maketitle

%%%SECTION%%%%%%%%%%%%
\section{Introduction}
Ce document décrit les différents éléments de la suite logicielle GBMProject3A.
Le code fourni est écrit en C++ avec comme unique dépendance l'API Qt.
Il se compose d'un ensemble de classes qui permettent:
\begin{itemize}
\item de communiquer avec un système Arduino que ce soit en réception ou en émission 
\item de traiter des signaux (filtrage, FFT)
\item d'afficher des signaux
\item de faire des requêtes sur une base de données MySQL
\end{itemize}

Une série d'exemples permet d'implémenter chaque 

%%%SECTION%%%%%%%%%%%%
\section{Exemple 1: Récupérer des données depuis un Arduino (liaison filaire)}

%%%SECTION%%%%%%%%%%%%
\section{Exemple 2: : Récupérer des données depuis un Arduino (liaison Bluetooth)}

%\begin{figure}
% \centering
%    \includegraphics[width=10cm]{Exemple2.jpg}
%    \caption{Exemple 2}
%    \label{Exemple2}
%\end{figure}



\end{document}