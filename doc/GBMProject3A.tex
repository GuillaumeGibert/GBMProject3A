\documentclass{article}

\usepackage{graphicx}
\usepackage{amsmath}
\usepackage{amssymb}
\usepackage{gensymb}

\begin{document}

\title{Documentation de la suite logicielle GBMProject3A}
\author{Guillaume Gibert}
\maketitle

%%%SECTION%%%%%%%%%%%%
\section{Introduction}
Ce document décrit les différents éléments de la suite logicielle GBMProject3A.

\subsection{Côté PC}
Le code fourni est écrit en C++ avec comme unique dépendance l'API Qt.
Il se compose d'un ensemble de classes qui permettent:
\begin{itemize}
\item de communiquer avec un système Arduino que ce soit en réception ou en émission;
\item de traiter des signaux (filtrage, FFT);
\item d'afficher des signaux;
\item de faire des requêtes sur une base de données MySQL.
\end{itemize}

\subsection{Côté Arduino}
Un code de test est fourni qui permet d'envoyer sous forme de chaîne de caractères standardisée les valeurs lues sur les 5 canaux analogiques de la carte Arduino. Une fonction de lecture d'évènements est également mise en place pour lire, en parallèle, les données sur le port série provenant du PC.


Une série d'exemples permet d'implémenter chaque fonctionnalité. Ils vont être décrit dans la suite du document.

%%%SECTION%%%%%%%%%%%%
\section{Exemple 1: Récupérer des données depuis un Arduino (liaison filaire)}

%%%SECTION%%%%%%%%%%%%
\section{Exemple 2: Récupérer des données depuis un Arduino (liaison Bluetooth)}

%\begin{figure}
% \centering
%    \includegraphics[width=10cm]{Exemple2.jpg}
%    \caption{Exemple 2}
%    \label{Exemple2}
%\end{figure}

%%%SECTION%%%%%%%%%%%%
\section{Exemple 3: Récupérer des données depuis un générateur de signaux}

%%%SECTION%%%%%%%%%%%%
\section{Exemple 4: Afficher des signaux temporels}

%%%SECTION%%%%%%%%%%%%
\section{Exemple 5: Filtrer des signaux temporels}


%%%SECTION%%%%%%%%%%%%
\section{Exemple 6: Calculer une transformée de Fourier rapide (FFT)}


%%%SECTION%%%%%%%%%%%%
\section{Exemple 7:  Envoyer une chaîne de caractères à l'Arduino}

%%%SECTION%%%%%%%%%%%%
\section{Exemple 8:  Gérer une base de données "patient" depuis une interface graphique}

\end{document}